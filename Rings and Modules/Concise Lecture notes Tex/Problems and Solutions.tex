\documentclass[lec]{subfiles}

\begin{document}
\chapter*{Problems and Solutions} %Set chapter name
\addcontentsline{toc}{chapter}{Problems and Solutions} %Set chapter title
\setcounter{chapter}{2} %Set chapter counter
\setcounter{section}{0}
%The content
\section{Lecture-2}

\begin{tcolorbox}
$R$ be a ring with unity. $a$ has right inverse and no left inverse. Show that it has infinite many right inverse.
\end{tcolorbox}
\textit{Solution.} Let $b$ be a right-inverse of $a$. For any $i \geq 0$, we define $b_i = (1-ba)a^i + b$. Show that if $a$ doesn't have a left-inverse, the $b_i$ are pairwise distinct right-inverses of $a$. 

\begin{tcolorbox}
    $1+xy \in R^* \implies 1 +yx \in R^*$
\end{tcolorbox}
\textit{Solution.}  Interprete this identity is by generalizing it:

$$(\lambda-ba)^{-1}=\lambda^{-1}+\lambda^{-1}b(\lambda-ab)^{-1}a.\qquad\qquad\qquad(*)$$

Note that this is both more general than the original formulation (set $\lambda=1$) and equivalent to it (rescale). Now the geometric series argument makes perfect sense in the ring $R((\lambda^{-1}))$ of formal Laurent power series, where $R$ is the original ring or even the "universal ring" $\mathbb{Z}\langle a,b\rangle:$

$$ (\lambda-ba)^{-1}=\lambda^{-1}+\sum_{n\geq 1}\lambda^{-n-1}(ba)^n=\lambda^{-1}(1+\sum_{n\geq 0}\lambda^{-n-1}b(ab)^n a)=\lambda^{-1}(1+b(\lambda-ab)^{-1}a). $$

For $\lambda = 1 , a = x, b = -y$ we can get our desired result. 

\begin{tcolorbox}
    \[
    \Q[\sqrt{d}] \cup \bar{\Z}= 
\begin{cases}
    \Z\left[\frac{\sqrt{d}+1}{2}\right],& \text{if } d \equiv 1 \pmod{4}\\
    \Z[\sqrt{d}],              & \text{if } d \not \equiv 1 \pmod{4}
\end{cases}
\]
\end{tcolorbox}
\textit{Proof.} An element of Algebraic Integral ring is called \textbf{Integral element}. A integral element's($\alpha$) irreducible polynomial has integer coefficient $\iff$ $\alpha \in \bar{\Z}$. Notice that, $\sqrt{d}$ is Integral as it satisfy $x^2-d$.

\vspace{0.1cm}

\hspace{0.1cm} If $d\equiv 1 \bmod 4$, then the monic irreducible polynomial of $\left(\frac{\sqrt{d}+1}{2}\right)$ over $\mathbb{Q}$ is $x^2 -x + \frac{(1-d)}{4}$ which is in $\mathbb{Z}[x]$, so $\left[\frac{\sqrt{d}+1}{2}\right]$ is integral. Thus the integral closure of $\mathbb{Z}$ in $\mathbb{Q}(\sqrt{d})$ contains the subring $\mathbb{Z}[\sqrt{d}]$, and the subring $\mathbb{Z}\left[\frac{\sqrt{d}+1}{2}\right]$ if $d\equiv 1 \bmod 4$. We will show that there are no other integral elements.

\vspace{0.2cm}

\hspace{0.2cm} An element $a+b\sqrt{d}$ with rational $a$ and $b\neq0$ is integral iff its monic irreducible polynomial $x^2 -2ax +(a^2 -db^2)$ belongs to $\mathbb{Z}[x]$. Therefore, $2a$,$2b$ are integers. If $a=\frac{(2k+1)}{2}$, for $k\in\mathbb{Z}$, then it is easy to see that $a^2 - db^2 \in \mathbb{Z}$ iff $b=\frac{2l+1}{2}$ for some $l\in\mathbb{Z}$, and $(2k+1)^2 - d(2l+1)^2$ is divisible by $4$. The latter implies that  $d\equiv 1 \bmod 4$. In turn, if $d\equiv 1 \bmod 4$ then every element $\frac{2k+1}{2} +\left(\frac{2l+1}{2}\right)\sqrt{d}$ is integral.

Thus, integral elements of $\mathbb{Q}(\sqrt{d})$ are equal to $\mathbb{Z}[\sqrt{d}]$ if $d\not \equiv 1 \bmod 4$, and $\Z\left[\frac{\sqrt{d}+1}{2}\right]$ if $d\equiv 1 \bmod 4$.



\end{document}